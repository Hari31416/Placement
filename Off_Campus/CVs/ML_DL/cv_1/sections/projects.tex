%-----------PROJECTS--------------------------
% Use resumeQuadHeading if four elements are feasible (ex: demo video link), else use resumeTrioHeading. Keep the bullet points simple and concise and try to cover wide variety of skills you have used to build these projects

\section{Projects}
\resumeHeadingListStart{}

% House Prices - Advanced Regression Techniques
\resumeTrioHeading{House Prices Prediction}{Python, pandas, scikit-learn, kaggle, Matplotlib, Seaborn}{\href{https://github.com/Hari31416/Portfolio/tree/main/ML/Housing_Price}{\uline{Source Code}}}
\resumeItemListStart{}
\resumeItem{Analyzed over \textbf{80} features to predict house prices using machine learning.}
\resumeItem{Performed \textbf{data visualization} and \textbf{feature engineering} using Matplotlib and Seaborn, respectively.}
\resumeItem{Trained \textbf{multiple models} using scikit-learn and selected the best one by applying \textbf{grid search} and \textbf{cross-validation}. Achieved a \textbf{top 12\%} ranking on the Kaggle leaderboard.}
\resumeItemListEnd{}

% Digit Recognizer
\resumeTrioHeading{Digit Recognizer}{Python, TensorFlow, Keras, Kaggle}{\href{https://github.com/Hari31416/Portfolio/tree/main/DL/MNIST_Digits}{\uline{Source Code}}}
\resumeItemListStart{}
\resumeItem{Developed a very deep \textbf{convolutional neural network} using TensorFlow and Keras with \textbf{dropout} and \textbf{batch normalization} to improve performance.}
\resumeItem{Achieved an accuracy of \textbf{99.48\%} on the test set, securing a place in the \textbf{top 15\%} on the Kaggle leaderboard.}
\resumeItemListEnd{}


% Food Vision
\resumeTrioHeading{Food Vision}{Python, TensorFlow, Colab}{\href{https://github.com/Hari31416/Portfolio/tree/main/DL/Food\%20Vision}{\uline{Source Code}}}
\resumeItemListStart{}
\resumeItem{Developed a deep \textbf{neural network} using TensorFlow and Keras to classify \textbf{101 categories of food}.}
\resumeItem{Used a pretrained \textbf{EfficientNet} model to extract features from the food images, and then \textbf{fine-tuned} the model to improve its accuracy.}
\resumeItem{Achieved an accuracy of \textbf{80\%} on the test set, demonstrating the effectiveness of the approach in addressing complex image recognition problems.}
\resumeItemListEnd{}


% Natural Language Processing with Disaster Tweets
\resumeTrioHeading{NLP With Disaster Tweets}{Python, TensorFlow, NLP, Text Vectorization, LSTM, GRU, CNN}{\href{https://github.com/Hari31416/Portfolio/tree/main/ML/Disaster_Tweets}{\uline{Source Code}}}
\resumeItemListStart{}
\resumeItem{Developed NLP models to classify disaster and non-disaster tweets using \textbf{text vectorization}, various \textbf{word embeddings}, and deep learning models including \textbf{LSTM}, \textbf{GRU}, and \textbf{1D CNNs}}
\resumeItem{Utilized the \textbf{Universal Sentence Encoder} to create embeddings on both the character and word levels, and implemented a \textbf{multivariate} model using the \textbf{functional API} of \textbf{TensorFlow}.}
\resumeItemListEnd{}

\resumeHeadingListEnd{}
